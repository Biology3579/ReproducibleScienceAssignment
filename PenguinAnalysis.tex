% Options for packages loaded elsewhere
\PassOptionsToPackage{unicode}{hyperref}
\PassOptionsToPackage{hyphens}{url}
%
\documentclass[
]{article}
\usepackage{amsmath,amssymb}
\usepackage{iftex}
\ifPDFTeX
  \usepackage[T1]{fontenc}
  \usepackage[utf8]{inputenc}
  \usepackage{textcomp} % provide euro and other symbols
\else % if luatex or xetex
  \usepackage{unicode-math} % this also loads fontspec
  \defaultfontfeatures{Scale=MatchLowercase}
  \defaultfontfeatures[\rmfamily]{Ligatures=TeX,Scale=1}
\fi
\usepackage{lmodern}
\ifPDFTeX\else
  % xetex/luatex font selection
    \setmainfont[]{Arial}
\fi
% Use upquote if available, for straight quotes in verbatim environments
\IfFileExists{upquote.sty}{\usepackage{upquote}}{}
\IfFileExists{microtype.sty}{% use microtype if available
  \usepackage[]{microtype}
  \UseMicrotypeSet[protrusion]{basicmath} % disable protrusion for tt fonts
}{}
\makeatletter
\@ifundefined{KOMAClassName}{% if non-KOMA class
  \IfFileExists{parskip.sty}{%
    \usepackage{parskip}
  }{% else
    \setlength{\parindent}{0pt}
    \setlength{\parskip}{6pt plus 2pt minus 1pt}}
}{% if KOMA class
  \KOMAoptions{parskip=half}}
\makeatother
\usepackage{xcolor}
\usepackage[left = 0.5cm, right = 1cm, top = 1cm, bottom =
1cm]{geometry}
\usepackage{color}
\usepackage{fancyvrb}
\newcommand{\VerbBar}{|}
\newcommand{\VERB}{\Verb[commandchars=\\\{\}]}
\DefineVerbatimEnvironment{Highlighting}{Verbatim}{commandchars=\\\{\}}
% Add ',fontsize=\small' for more characters per line
\usepackage{framed}
\definecolor{shadecolor}{RGB}{248,248,248}
\newenvironment{Shaded}{\begin{snugshade}}{\end{snugshade}}
\newcommand{\AlertTok}[1]{\textcolor[rgb]{0.94,0.16,0.16}{#1}}
\newcommand{\AnnotationTok}[1]{\textcolor[rgb]{0.56,0.35,0.01}{\textbf{\textit{#1}}}}
\newcommand{\AttributeTok}[1]{\textcolor[rgb]{0.13,0.29,0.53}{#1}}
\newcommand{\BaseNTok}[1]{\textcolor[rgb]{0.00,0.00,0.81}{#1}}
\newcommand{\BuiltInTok}[1]{#1}
\newcommand{\CharTok}[1]{\textcolor[rgb]{0.31,0.60,0.02}{#1}}
\newcommand{\CommentTok}[1]{\textcolor[rgb]{0.56,0.35,0.01}{\textit{#1}}}
\newcommand{\CommentVarTok}[1]{\textcolor[rgb]{0.56,0.35,0.01}{\textbf{\textit{#1}}}}
\newcommand{\ConstantTok}[1]{\textcolor[rgb]{0.56,0.35,0.01}{#1}}
\newcommand{\ControlFlowTok}[1]{\textcolor[rgb]{0.13,0.29,0.53}{\textbf{#1}}}
\newcommand{\DataTypeTok}[1]{\textcolor[rgb]{0.13,0.29,0.53}{#1}}
\newcommand{\DecValTok}[1]{\textcolor[rgb]{0.00,0.00,0.81}{#1}}
\newcommand{\DocumentationTok}[1]{\textcolor[rgb]{0.56,0.35,0.01}{\textbf{\textit{#1}}}}
\newcommand{\ErrorTok}[1]{\textcolor[rgb]{0.64,0.00,0.00}{\textbf{#1}}}
\newcommand{\ExtensionTok}[1]{#1}
\newcommand{\FloatTok}[1]{\textcolor[rgb]{0.00,0.00,0.81}{#1}}
\newcommand{\FunctionTok}[1]{\textcolor[rgb]{0.13,0.29,0.53}{\textbf{#1}}}
\newcommand{\ImportTok}[1]{#1}
\newcommand{\InformationTok}[1]{\textcolor[rgb]{0.56,0.35,0.01}{\textbf{\textit{#1}}}}
\newcommand{\KeywordTok}[1]{\textcolor[rgb]{0.13,0.29,0.53}{\textbf{#1}}}
\newcommand{\NormalTok}[1]{#1}
\newcommand{\OperatorTok}[1]{\textcolor[rgb]{0.81,0.36,0.00}{\textbf{#1}}}
\newcommand{\OtherTok}[1]{\textcolor[rgb]{0.56,0.35,0.01}{#1}}
\newcommand{\PreprocessorTok}[1]{\textcolor[rgb]{0.56,0.35,0.01}{\textit{#1}}}
\newcommand{\RegionMarkerTok}[1]{#1}
\newcommand{\SpecialCharTok}[1]{\textcolor[rgb]{0.81,0.36,0.00}{\textbf{#1}}}
\newcommand{\SpecialStringTok}[1]{\textcolor[rgb]{0.31,0.60,0.02}{#1}}
\newcommand{\StringTok}[1]{\textcolor[rgb]{0.31,0.60,0.02}{#1}}
\newcommand{\VariableTok}[1]{\textcolor[rgb]{0.00,0.00,0.00}{#1}}
\newcommand{\VerbatimStringTok}[1]{\textcolor[rgb]{0.31,0.60,0.02}{#1}}
\newcommand{\WarningTok}[1]{\textcolor[rgb]{0.56,0.35,0.01}{\textbf{\textit{#1}}}}
\usepackage{graphicx}
\makeatletter
\def\maxwidth{\ifdim\Gin@nat@width>\linewidth\linewidth\else\Gin@nat@width\fi}
\def\maxheight{\ifdim\Gin@nat@height>\textheight\textheight\else\Gin@nat@height\fi}
\makeatother
% Scale images if necessary, so that they will not overflow the page
% margins by default, and it is still possible to overwrite the defaults
% using explicit options in \includegraphics[width, height, ...]{}
\setkeys{Gin}{width=\maxwidth,height=\maxheight,keepaspectratio}
% Set default figure placement to htbp
\makeatletter
\def\fps@figure{htbp}
\makeatother
\setlength{\emergencystretch}{3em} % prevent overfull lines
\providecommand{\tightlist}{%
  \setlength{\itemsep}{0pt}\setlength{\parskip}{0pt}}
\setcounter{secnumdepth}{-\maxdimen} % remove section numbering
\ifLuaTeX
  \usepackage{selnolig}  % disable illegal ligatures
\fi
\usepackage{bookmark}
\IfFileExists{xurl.sty}{\usepackage{xurl}}{} % add URL line breaks if available
\urlstyle{same}
\hypersetup{
  pdftitle={PenguinsAnalysis},
  pdfauthor={Biology3579},
  hidelinks,
  pdfcreator={LaTeX via pandoc}}

\title{PenguinsAnalysis}
\author{Biology3579}
\date{2024-12-6}

\begin{document}
\maketitle

\subsubsection{Introduction}\label{introduction}

The genus Pygoscelis, which includes Adélie (\emph{Pygoscelis adeliae}),
Chinstrap (\emph{Pygoscelis antarcticus}), and Gentoo (\emph{Pygoscelis
papua}) penguins, occupies overlapping ranges across the Western
Antarctic Peninsula. Despite this sympatric distribution, these species
exhibit distinct feeding ecologies, ranging from specialist to
generalist strategies, allowing them to occupy different ecological
niches and minimising competition. Diet composition analysis of
Chinstrap penguins shows a high degree of specialization, with their
diet consisting almost exclusively of the krill species \emph{Euphausia
superba}. In contrast, while Adélie penguins also rely heavily on krill,
their diet is more varied, consisting of both \emph{E. superba} and
\emph{E. crystallorophias} krill species, along with notothenioid fish.
Dietary analyses of Gentoo penguins reveal that this species has the
most varied feeding habits, with a generalist diet consisting of a broad
range of crustaceans and fish. Interestingly, krill constitutes a
smaller portion of their diet compared to other \emph{Pygoscelis}
species, highlighting a distinct difference in the feeding ecology of
this species.

The bill morphology of penguins is a critical aspect of their feeding
ecology, as it directly influences their ability to capture and consume
different prey types. A study by Chávez-Hoffmeister (2020) found that
penguin species display distinct bill shapes that are closely tied to
their specific feeding strategies {[}{]}. For example, krill-eating
penguins tend to have wider beaks and broader jaws, which help them
efficiently filter and capture krill. In contrast, fish-eating species
possess narrower, more robust beaks designed to exert greater bite
force, allowing them to grasp and hold onto slippery fish more
effectively. These morphological adaptations are essential for the
species' feeding behavior, enabling them to exploit their preferred prey
types. As such, each penguin species requires bill morphology that is
finely tuned to its ecological niche, ensuring efficient foraging and
survival in their specific environments.

\subsubsection{Hypothesis}\label{hypothesis}

Based on the importance of bill morphology in feeding behaviors and the
differences in feeding ecologies between the species, I hypothesize that
the three penguin species will exhibit distinct bill morphologies that
reflect their specific feeding strategies. Specifically, since Gentoo
penguins have a much more diverse diet compared to the other two
species, I expect them to exhibit the most distinct bill morphology.

\subsubsection{Methods}\label{methods}

\paragraph{Data}\label{data}

To investigate the differences in bill morphology across penguin
species, I will be utilizing the palmerpenguins dataset, which is
derived from a study conducted by Gorman et al.~(2014) on Pygoscelis
penguins in the Palmer Archipelago, located west of the Antarctic
Peninsula, near Anvers Island. The data were collected from 2007 to 2009
across three islands: Biscoe (64° 48.9'S, 63° 46.9'W), Torgersen (64°
46.9'S, 64° 04.9'W), and Dream (64° 43.9'S, 64° 13.9'W). This dataset
includes multiple measurements taken on three species of Pygoscelis
penguins---Adélie, Chinstrap, and Gentoo---observed on these three
islands, including body measurements (culmen length, culmen depth,
flipper length, and body mass), sex information, nesting details
(including clutch completion and egg laying dates), and stable isotope
data (Δ15N and Δ13C) for 344 individual penguins across three species.

\paragraph{Statistical analysis}\label{statistical-analysis}

To examine the differences in bill morphology across the three penguin
speciesI will focus on two key measurements: bill length and bill depth.
These measurements serve as indicators of bill morphology, providing
insights into how adaptations to feeding strategies differ among species
with varying diets.

\begin{enumerate}
\def\labelenumi{\arabic{enumi}.}
\tightlist
\item
  Ratio of Bill Length to Bill Depth The first step in my analysis will
  involve examining the ratio of bill length to bill depth for each
  species. This ratio is a useful indicator of overall bill shape and
  can help highlight differences in bill morphology that are associated
  with each species' feeding ecology. For example, a more specialized
  feeder may have a differently proportioned bill compared to a more
  generalist species.
\end{enumerate}

ANOVA tests assess whether there are any statistically significant
differences in bill morphology, bill length, or bill depth between the
species. The Tukey-Kramer test provides pairwise comparisons to identify
which species pairs differ significantly from each other.

Statistical Test: One-Way ANOVA I will use one-way ANOVA to compare the
mean ratio of bill length to bill depth across the three species
(Adélie, Chinstrap, and Gentoo). ANOVA is appropriate for this analysis
because I am comparing the means of a continuous variable (bill ratio)
across more than two groups (species).

Rationale for ANOVA: ANOVA is ideal for testing if there are significant
differences in the means of the bill ratio across the three species. If
the ANOVA reveals a significant result, this would indicate that at
least one species has a significantly different bill morphology in terms
of its shape (ratio of length to depth).

Post-Hoc Analysis: If the ANOVA test shows significant differences, I
will conduct post-hoc tests, such as Tukey's HSD (Honest Significant
Difference), to determine which specific species pairs (e.g., Adélie
vs.~Gentoo) show significant differences in bill morphology. This is
necessary to identify exactly where the differences lie between species.

\begin{enumerate}
\def\labelenumi{\arabic{enumi}.}
\setcounter{enumi}{1}
\tightlist
\item
  Bill Length and Bill Depth Comparisons Following the ratio analysis, I
  will delve deeper into the individual measurements of bill length and
  bill depth. This will allow for a more detailed comparison of these
  specific morphological features across the species.
\end{enumerate}

Statistical Test: One-Way ANOVA (for normally distributed data) I will
first assess the normality of bill length and bill depth data for each
species using tests such as the Shapiro-Wilk test. If the data are
normally distributed, I will use one-way ANOVA to compare the means of
bill length and bill depth across the three species.

Rationale for ANOVA: ANOVA is again used here to determine if there are
significant differences in the bill length and bill depth among the
species. A significant result would indicate that the species exhibit
different morphologies in these specific aspects, which could be linked
to their feeding behavior and ecological niches. I used this dataset to
examine the culmen shapes of the different species, specifically, I will
be focusing on two measurements - bill length and bill depth - as
indicators of bill morphology.

To provide more precise estimates of the species' bill characteristics,
I also calculated 95\% confidence intervals for both bill length and
bill depth. Confidence intervals are important because they offer a
range within which the true population mean is likely to lie, allowing
for a better understanding of the variability and uncertainty around the
estimates. By including these intervals, I am able to give a more robust
and complete picture of the data, helping to interpret the significance
of the differences observed in the ANOVA and Tukey-Kramer tests.

\subsection{Analysis}\label{analysis}

\begin{Shaded}
\begin{Highlighting}[]
\CommentTok{\#load all the required packages}
\FunctionTok{library}\NormalTok{(tidyverse) }\CommentTok{\#for cleaning the data}
\FunctionTok{library}\NormalTok{(janitor) }\CommentTok{\#for cleaning the data}
\FunctionTok{library}\NormalTok{(palmerpenguins) }\CommentTok{\#contains the dataset}
\FunctionTok{library}\NormalTok{(here) }\CommentTok{\#to specify the directory}
\FunctionTok{library}\NormalTok{(ggplot2) }\CommentTok{\#for making plots}
\FunctionTok{library}\NormalTok{(lme4) }\CommentTok{\#for making linear models}
\FunctionTok{library}\NormalTok{(grid) }\CommentTok{\#for making multi{-}panel figures}
\FunctionTok{library}\NormalTok{(gridExtra) }\CommentTok{\#for making multi{-}panel figures}
\FunctionTok{library}\NormalTok{(dplyr) }\CommentTok{\#for manipulating the data}
\FunctionTok{library}\NormalTok{(ggsignif) }\CommentTok{\#for adding significance stars to plots}
\end{Highlighting}
\end{Shaded}

\begin{Shaded}
\begin{Highlighting}[]
\CommentTok{\#Load the raw data and save it}

\FunctionTok{write.csv}\NormalTok{(penguins\_raw, }\FunctionTok{here}\NormalTok{(}\StringTok{"data"}\NormalTok{, }\StringTok{"penguins\_raw.csv"}\NormalTok{)) }\CommentTok{\#To write data to csv}
\NormalTok{penguins\_raw }\OtherTok{\textless{}{-}} \FunctionTok{read.csv}\NormalTok{(}\FunctionTok{here}\NormalTok{(}\StringTok{"data"}\NormalTok{,}\StringTok{"penguins\_raw.csv"}\NormalTok{)) }\CommentTok{\#Load data }

\FunctionTok{glimpse}\NormalTok{(penguins\_raw) }\CommentTok{\# Quick summary of the structure of the raw data}
\end{Highlighting}
\end{Shaded}

\begin{verbatim}
## Rows: 344
## Columns: 18
## $ X                   <int> 1, 2, 3, 4, 5, 6, 7, 8, 9, 10, 11, 12, 13, 14, 15,~
## $ studyName           <chr> "PAL0708", "PAL0708", "PAL0708", "PAL0708", "PAL07~
## $ Sample.Number       <int> 1, 2, 3, 4, 5, 6, 7, 8, 9, 10, 11, 12, 13, 14, 15,~
## $ Species             <chr> "Adelie Penguin (Pygoscelis adeliae)", "Adelie Pen~
## $ Region              <chr> "Anvers", "Anvers", "Anvers", "Anvers", "Anvers", ~
## $ Island              <chr> "Torgersen", "Torgersen", "Torgersen", "Torgersen"~
## $ Stage               <chr> "Adult, 1 Egg Stage", "Adult, 1 Egg Stage", "Adult~
## $ Individual.ID       <chr> "N1A1", "N1A2", "N2A1", "N2A2", "N3A1", "N3A2", "N~
## $ Clutch.Completion   <chr> "Yes", "Yes", "Yes", "Yes", "Yes", "Yes", "No", "N~
## $ Date.Egg            <chr> "2007-11-11", "2007-11-11", "2007-11-16", "2007-11~
## $ Culmen.Length..mm.  <dbl> 39.1, 39.5, 40.3, NA, 36.7, 39.3, 38.9, 39.2, 34.1~
## $ Culmen.Depth..mm.   <dbl> 18.7, 17.4, 18.0, NA, 19.3, 20.6, 17.8, 19.6, 18.1~
## $ Flipper.Length..mm. <int> 181, 186, 195, NA, 193, 190, 181, 195, 193, 190, 1~
## $ Body.Mass..g.       <int> 3750, 3800, 3250, NA, 3450, 3650, 3625, 4675, 3475~
## $ Sex                 <chr> "MALE", "FEMALE", "FEMALE", NA, "FEMALE", "MALE", ~
## $ Delta.15.N..o.oo.   <dbl> NA, 8.94956, 8.36821, NA, 8.76651, 8.66496, 9.1871~
## $ Delta.13.C..o.oo.   <dbl> NA, -24.69454, -25.33302, NA, -25.32426, -25.29805~
## $ Comments            <chr> "Not enough blood for isotopes.", NA, NA, "Adult n~
\end{verbatim}

\begin{Shaded}
\begin{Highlighting}[]
\CommentTok{\#Clean the raw data and save it separately}
\FunctionTok{source}\NormalTok{(}\FunctionTok{here}\NormalTok{(}\StringTok{"functions"}\NormalTok{, }\StringTok{"cleaning\_and\_curating.R"}\NormalTok{))}
\NormalTok{penguins\_clean }\OtherTok{\textless{}{-}} \FunctionTok{cleaning\_penguins}\NormalTok{(penguins\_raw)}
\FunctionTok{write\_csv}\NormalTok{(penguins\_clean, }\FunctionTok{here}\NormalTok{(}\StringTok{"data"}\NormalTok{, }\StringTok{"penguins\_clean.csv"}\NormalTok{))}
\FunctionTok{glimpse}\NormalTok{(penguins\_clean) }\CommentTok{\# Quick summary of clean data}
\end{Highlighting}
\end{Shaded}

\begin{verbatim}
## Rows: 344
## Columns: 18
## $ x                 <int> 1, 2, 3, 4, 5, 6, 7, 8, 9, 10, 11, 12, 13, 14, 15, 1~
## $ study_name        <chr> "PAL0708", "PAL0708", "PAL0708", "PAL0708", "PAL0708~
## $ sample_number     <int> 1, 2, 3, 4, 5, 6, 7, 8, 9, 10, 11, 12, 13, 14, 15, 1~
## $ species           <chr> "Adelie", "Adelie", "Adelie", "Adelie", "Adelie", "A~
## $ region            <chr> "Anvers", "Anvers", "Anvers", "Anvers", "Anvers", "A~
## $ island            <chr> "Torgersen", "Torgersen", "Torgersen", "Torgersen", ~
## $ stage             <chr> "Adult, 1 Egg Stage", "Adult, 1 Egg Stage", "Adult, ~
## $ individual_id     <chr> "N1A1", "N1A2", "N2A1", "N2A2", "N3A1", "N3A2", "N4A~
## $ clutch_completion <chr> "Yes", "Yes", "Yes", "Yes", "Yes", "Yes", "No", "No"~
## $ date_egg          <chr> "2007-11-11", "2007-11-11", "2007-11-16", "2007-11-1~
## $ culmen_length_mm  <dbl> 39.1, 39.5, 40.3, NA, 36.7, 39.3, 38.9, 39.2, 34.1, ~
## $ culmen_depth_mm   <dbl> 18.7, 17.4, 18.0, NA, 19.3, 20.6, 17.8, 19.6, 18.1, ~
## $ flipper_length_mm <int> 181, 186, 195, NA, 193, 190, 181, 195, 193, 190, 186~
## $ body_mass_g       <int> 3750, 3800, 3250, NA, 3450, 3650, 3625, 4675, 3475, ~
## $ sex               <chr> "MALE", "FEMALE", "FEMALE", NA, "FEMALE", "MALE", "F~
## $ delta_15_n_o_oo   <dbl> NA, 8.94956, 8.36821, NA, 8.76651, 8.66496, 9.18718,~
## $ delta_13_c_o_oo   <dbl> NA, -24.69454, -25.33302, NA, -25.32426, -25.29805, ~
## $ comments          <chr> "Not enough blood for isotopes.", NA, NA, "Adult not~
\end{verbatim}

\begin{Shaded}
\begin{Highlighting}[]
\CommentTok{\#Curate the clean data and save it separately}
\FunctionTok{source}\NormalTok{(}\FunctionTok{here}\NormalTok{(}\StringTok{"functions"}\NormalTok{, }\StringTok{"cleaning\_and\_curating.R"}\NormalTok{))}
\NormalTok{analysis\_data }\OtherTok{\textless{}{-}} \FunctionTok{curating\_penguins\_clean}\NormalTok{(penguins\_clean)}
\FunctionTok{write\_csv}\NormalTok{(analysis\_data, }\FunctionTok{here}\NormalTok{(}\StringTok{"data"}\NormalTok{, }\StringTok{"analysis\_data.csv"}\NormalTok{))}

\FunctionTok{glimpse}\NormalTok{(analysis\_data) }\CommentTok{\# Quick summary of analysis data}
\end{Highlighting}
\end{Shaded}

\begin{verbatim}
## Rows: 333
## Columns: 9
## $ x               <int> 1, 2, 3, 5, 6, 7, 8, 13, 14, 15, 16, 17, 18, 19, 20, 2~
## $ species         <fct> Adelie, Adelie, Adelie, Adelie, Adelie, Adelie, Adelie~
## $ island          <chr> "Torgersen", "Torgersen", "Torgersen", "Torgersen", "T~
## $ individual_id   <chr> "N1A1", "N1A2", "N2A1", "N3A1", "N3A2", "N4A1", "N4A2"~
## $ bill_length_mm  <dbl> 39.1, 39.5, 40.3, 36.7, 39.3, 38.9, 39.2, 41.1, 38.6, ~
## $ bill_depth_mm   <dbl> 18.7, 17.4, 18.0, 19.3, 20.6, 17.8, 19.6, 17.6, 21.2, ~
## $ body_mass_g     <dbl> 3750, 3800, 3250, 3450, 3650, 3625, 4675, 3200, 3800, ~
## $ sex             <fct> male, female, female, female, male, female, male, fema~
## $ bill_morphology <dbl> 2.090909, 2.270115, 2.238889, 1.901554, 1.907767, 2.18~
\end{verbatim}

\subsubsection{Exploratory analysis}\label{exploratory-analysis}

\begin{center}\includegraphics{PenguinAnalysis_files/figure-latex/bad-figure-plot-1} \end{center}

\emph{Figure 1: A bar chart of mean bill length vs.~mean bill depth for
three species of Pygoscelis} penguins

\begin{enumerate}
\def\labelenumi{\alph{enumi})}
\setcounter{enumi}{1}
\tightlist
\item
  Write about how your design choices mislead the reader about the
  underlying data (200-300 words).
\end{enumerate}

This figure displays the mean bill length and depth for Adelie,
Chinstrap, and Gentoo penguins. However, several design choices in this
figure mislead the reader about the underlying data:

\begin{itemize}
\item
  A bar chart is a poor choice for displaying relationships between two
  continuous variables. Bar charts are typically used for categorical
  data, and applying them to continuous data (such as bill length and
  depth) oversimplifies the relationship. A scatter plot would be much
  more suited for representing .. {[}{]}
\item
  The figure only shows the summary statistics (mean values) for each
  species but does not include any raw data points or an indication of
  the data distribution. Without this, the reader cannot assess
  variability, outliers, or trends within each species, which are
  critical for understanding the data. This omission hides the true
  complexity and spread of the data. {[}{]}
\item
  The plot also lacks error bars or any indication of variability, such
  as standard deviation or confidence intervals. These are essential for
  understanding the precision of the mean values and the reliability of
  the comparisons between species. Without this information, the plot
  may give the false impression that the differences between species are
  more significant than they actually are {[}{]}
\item
  The colors used for the three species are all very similar and
  difficult to distinguish at a glance. This makes it hard for the
  viewer to quickly differentiate between species, even with a legend,
  and could cause confusion. More contrasting colors or patterns should
  have been used to make the distinction clearer.
\item
  The axis labels are not informative and do not covey the units.
  Clearer labels would indicate what is being measured and \ldots{}
\end{itemize}

\textbf{Good exploratory figure}

\begin{Shaded}
\begin{Highlighting}[]
\FunctionTok{source}\NormalTok{(}\FunctionTok{here}\NormalTok{(}\StringTok{"functions"}\NormalTok{, }\StringTok{"plotting.R"}\NormalTok{))}
\NormalTok{exploratory\_figure }\OtherTok{\textless{}{-}} \FunctionTok{plot\_exploratory\_figure}\NormalTok{(analysis\_data)}
\NormalTok{exploratory\_figure }\CommentTok{\# Show the plot}
\end{Highlighting}
\end{Shaded}

\begin{center}\includegraphics{PenguinAnalysis_files/figure-latex/exploratory-figure-plot-1} \end{center}

\begin{Shaded}
\begin{Highlighting}[]
\CommentTok{\#Save plot }
\CommentTok{\#gsave("exploratory\_figure.png", }
       \CommentTok{\#plot = exploratory\_figure, }
       \CommentTok{\#width = 8, height = 6, units = "in", dpi = 300)  }
\end{Highlighting}
\end{Shaded}

\emph{Figure 2: Scatterplot showing the distribution of bill lengths and
bill depths for three species of Pygoscelis} penguins

Figure 2.A, we observe that the distribution of bill length and bill
depth varies between the three penguin species. Adelie penguins tend to
have smaller bill lengths and depths compared to the other species,
while Chinstrap penguins show intermediate values. Gentoo penguins, on
the other hand, display larger bills, with both the length and depth
showing more variability across individuals\ldots{}

The dsitrbution ,,,

Howver to fully deterine whetehr these differeces are we \ldots{}

\subsubsection{Statistical analysis}\label{statistical-analysis-1}

I first carried out an ANOVA, followed by Tukey-Kramer tests, to
determine the general differences in bill morphology, using the bill
length-to-depth ratio as a proxy for this. I then further investigated
specific differences by testing bill length and bill depth separately to
understand the variation between species. To provide more precise
estimates of the species' bill characteristics, I also calculated 95\%
confidence intervals for both bill length and bill depth.

\paragraph{Differences in bill
morphology}\label{differences-in-bill-morphology}

\begin{Shaded}
\begin{Highlighting}[]
\CommentTok{\# ANOVA to test for differences in bill morphology between penguin species}
\NormalTok{anova\_morphology }\OtherTok{\textless{}{-}} \FunctionTok{aov}\NormalTok{(bill\_morphology }\SpecialCharTok{\textasciitilde{}}\NormalTok{ species, }\AttributeTok{data =}\NormalTok{ analysis\_data)}
\FunctionTok{summary}\NormalTok{(anova\_morphology)}
\end{Highlighting}
\end{Shaded}

\begin{verbatim}
##              Df Sum Sq Mean Sq F value Pr(>F)    
## species       2  73.17   36.59    1452 <2e-16 ***
## Residuals   330   8.32    0.03                   
## ---
## Signif. codes:  0 '***' 0.001 '**' 0.01 '*' 0.05 '.' 0.1 ' ' 1
\end{verbatim}

\begin{Shaded}
\begin{Highlighting}[]
\CommentTok{\# Tukey{-}Kramer test for pairwise comparisons of bill morphology between penguin species}
\NormalTok{tukey\_results\_morphology }\OtherTok{\textless{}{-}} \FunctionTok{TukeyHSD}\NormalTok{(}\FunctionTok{aov}\NormalTok{(bill\_morphology }\SpecialCharTok{\textasciitilde{}}\NormalTok{ species, }\AttributeTok{data =}\NormalTok{ analysis\_data))}
\FunctionTok{print}\NormalTok{(tukey\_results\_morphology)}
\end{Highlighting}
\end{Shaded}

\begin{verbatim}
##   Tukey multiple comparisons of means
##     95% family-wise confidence level
## 
## Fit: aov(formula = bill_morphology ~ species, data = analysis_data)
## 
## $species
##                       diff       lwr       upr p adj
## Chinstrap-Adelie 0.5322773 0.4774023 0.5871523     0
## Gentoo-Adelie    1.0551237 1.0089631 1.1012844     0
## Gentoo-Chinstrap 0.5228464 0.4660277 0.5796651     0
\end{verbatim}

\paragraph{Differences in bill length}\label{differences-in-bill-length}

\begin{Shaded}
\begin{Highlighting}[]
\CommentTok{\# ANOVA to test for differences in bill length between penguin species}
\NormalTok{anova\_length }\OtherTok{\textless{}{-}} \FunctionTok{aov}\NormalTok{(bill\_length\_mm }\SpecialCharTok{\textasciitilde{}}\NormalTok{ species, }\AttributeTok{data =}\NormalTok{ analysis\_data)}
\FunctionTok{summary}\NormalTok{(anova\_length)}
\end{Highlighting}
\end{Shaded}

\begin{verbatim}
##              Df Sum Sq Mean Sq F value Pr(>F)    
## species       2   7015    3508   397.3 <2e-16 ***
## Residuals   330   2914       9                   
## ---
## Signif. codes:  0 '***' 0.001 '**' 0.01 '*' 0.05 '.' 0.1 ' ' 1
\end{verbatim}

\begin{Shaded}
\begin{Highlighting}[]
\CommentTok{\# Tukey{-}Kramer test for pairwise comparisons of bill length between penguin species}
\NormalTok{tukey\_results\_length }\OtherTok{\textless{}{-}} \FunctionTok{TukeyHSD}\NormalTok{(}\FunctionTok{aov}\NormalTok{(bill\_length\_mm }\SpecialCharTok{\textasciitilde{}}\NormalTok{ species, }\AttributeTok{data =}\NormalTok{ analysis\_data))}
\FunctionTok{print}\NormalTok{(tukey\_results\_length)}
\end{Highlighting}
\end{Shaded}

\begin{verbatim}
##   Tukey multiple comparisons of means
##     95% family-wise confidence level
## 
## Fit: aov(formula = bill_length_mm ~ species, data = analysis_data)
## 
## $species
##                       diff       lwr        upr     p adj
## Chinstrap-Adelie 10.009851  8.982789 11.0369128 0.0000000
## Gentoo-Adelie     8.744095  7.880135  9.6080546 0.0000000
## Gentoo-Chinstrap -1.265756 -2.329197 -0.2023151 0.0148212
\end{verbatim}

\begin{Shaded}
\begin{Highlighting}[]
\CommentTok{\# Pipe to calculate 95\% confidence intervals for bill length}
\NormalTok{CIs\_bill\_length }\OtherTok{\textless{}{-}}\NormalTok{ analysis\_data }\SpecialCharTok{\%\textgreater{}\%}
  \FunctionTok{group\_by}\NormalTok{(species) }\SpecialCharTok{\%\textgreater{}\%}
  \FunctionTok{summarise}\NormalTok{(}
    \AttributeTok{mean\_length =} \FunctionTok{mean}\NormalTok{(bill\_length\_mm, }\AttributeTok{na.rm =} \ConstantTok{TRUE}\NormalTok{),}
    \AttributeTok{se\_length =} \FunctionTok{sd}\NormalTok{(bill\_length\_mm, }\AttributeTok{na.rm =} \ConstantTok{TRUE}\NormalTok{) }\SpecialCharTok{/} \FunctionTok{sqrt}\NormalTok{(}\FunctionTok{n}\NormalTok{()),}
    \AttributeTok{lower\_length\_ci =}\NormalTok{ mean\_length }\SpecialCharTok{{-}} \FloatTok{1.96} \SpecialCharTok{*}\NormalTok{ se\_length,}
    \AttributeTok{upper\_length\_ci =}\NormalTok{ mean\_length }\SpecialCharTok{+} \FloatTok{1.96} \SpecialCharTok{*}\NormalTok{ se\_length}
\NormalTok{  )}

\FunctionTok{print}\NormalTok{(CIs\_bill\_length)}
\end{Highlighting}
\end{Shaded}

\begin{verbatim}
## # A tibble: 3 x 5
##   species   mean_length se_length lower_length_ci upper_length_ci
##   <fct>           <dbl>     <dbl>           <dbl>           <dbl>
## 1 Adelie           38.8     0.220            38.4            39.3
## 2 Chinstrap        48.8     0.405            48.0            49.6
## 3 Gentoo           47.6     0.285            47.0            48.1
\end{verbatim}

\paragraph{Differences in bill depth}\label{differences-in-bill-depth}

\begin{Shaded}
\begin{Highlighting}[]
\CommentTok{\# ANOVA to test for differences in bill depth between penguin species}
\NormalTok{anova\_depth }\OtherTok{\textless{}{-}} \FunctionTok{aov}\NormalTok{(bill\_depth\_mm }\SpecialCharTok{\textasciitilde{}}\NormalTok{ species, }\AttributeTok{data =}\NormalTok{ analysis\_data)}
\FunctionTok{summary}\NormalTok{(anova\_depth)}
\end{Highlighting}
\end{Shaded}

\begin{verbatim}
##              Df Sum Sq Mean Sq F value Pr(>F)    
## species       2  870.8   435.4   344.8 <2e-16 ***
## Residuals   330  416.7     1.3                   
## ---
## Signif. codes:  0 '***' 0.001 '**' 0.01 '*' 0.05 '.' 0.1 ' ' 1
\end{verbatim}

\begin{Shaded}
\begin{Highlighting}[]
\CommentTok{\# Tukey{-}Kramer test for pairwise comparisons of bill depth between penguin species}
\NormalTok{tukey\_results\_depth }\OtherTok{\textless{}{-}} \FunctionTok{TukeyHSD}\NormalTok{(}\FunctionTok{aov}\NormalTok{(bill\_depth\_mm }\SpecialCharTok{\textasciitilde{}}\NormalTok{ species, }\AttributeTok{data =}\NormalTok{ analysis\_data))}
\FunctionTok{print}\NormalTok{(tukey\_results\_depth)}
\end{Highlighting}
\end{Shaded}

\begin{verbatim}
##   Tukey multiple comparisons of means
##     95% family-wise confidence level
## 
## Fit: aov(formula = bill_depth_mm ~ species, data = analysis_data)
## 
## $species
##                         diff       lwr        upr     p adj
## Chinstrap-Adelie  0.07332796 -0.315078  0.4617339 0.8968734
## Gentoo-Adelie    -3.35062162 -3.677347 -3.0238962 0.0000000
## Gentoo-Chinstrap -3.42394958 -3.826113 -3.0217860 0.0000000
\end{verbatim}

\begin{Shaded}
\begin{Highlighting}[]
\CommentTok{\# Pipe to calculate confidence intervals for bill depth}
\NormalTok{CIs\_bill\_depth }\OtherTok{\textless{}{-}}\NormalTok{ analysis\_data }\SpecialCharTok{\%\textgreater{}\%}
  \FunctionTok{group\_by}\NormalTok{(species) }\SpecialCharTok{\%\textgreater{}\%}
  \FunctionTok{summarise}\NormalTok{(}
    \AttributeTok{mean\_depth =} \FunctionTok{mean}\NormalTok{(bill\_depth\_mm, }\AttributeTok{na.rm =} \ConstantTok{TRUE}\NormalTok{),}
    \AttributeTok{se\_depth =} \FunctionTok{sd}\NormalTok{(bill\_depth\_mm, }\AttributeTok{na.rm =} \ConstantTok{TRUE}\NormalTok{) }\SpecialCharTok{/} \FunctionTok{sqrt}\NormalTok{(}\FunctionTok{n}\NormalTok{()),}
    \AttributeTok{lower\_depth\_ci =}\NormalTok{ mean\_depth }\SpecialCharTok{{-}} \FloatTok{1.96} \SpecialCharTok{*}\NormalTok{ se\_depth,}
    \AttributeTok{upper\_depth\_ci =}\NormalTok{ mean\_depth }\SpecialCharTok{+} \FloatTok{1.96} \SpecialCharTok{*}\NormalTok{ se\_depth}
\NormalTok{  )}

\FunctionTok{print}\NormalTok{(CIs\_bill\_depth)}
\end{Highlighting}
\end{Shaded}

\begin{verbatim}
## # A tibble: 3 x 5
##   species   mean_depth se_depth lower_depth_ci upper_depth_ci
##   <fct>          <dbl>    <dbl>          <dbl>          <dbl>
## 1 Adelie          18.3   0.101            18.1           18.5
## 2 Chinstrap       18.4   0.138            18.2           18.7
## 3 Gentoo          15.0   0.0904           14.8           15.2
\end{verbatim}

\#\#Results

\begin{Shaded}
\begin{Highlighting}[]
\FunctionTok{source}\NormalTok{(}\FunctionTok{here}\NormalTok{(}\StringTok{"functions"}\NormalTok{, }\StringTok{"plotting.R"}\NormalTok{))}
\NormalTok{results\_figure\_1 }\OtherTok{\textless{}{-}} \FunctionTok{results\_plot\_1}\NormalTok{(analysis\_data)}

\CommentTok{\#Save plot }
\CommentTok{\#gsave("exploratory\_figure.png", }
       \CommentTok{\#plot = exploratory\_figure, }
       \CommentTok{\#width = 8, height = 6, units = "in", dpi = 300)}
       
\NormalTok{results\_figure\_2 }\OtherTok{\textless{}{-}} \FunctionTok{results\_plot\_2}\NormalTok{(analysis\_data)}

\CommentTok{\#Save plot }
\CommentTok{\#gsave("exploratory\_figure.png", }
       \CommentTok{\#plot = exploratory\_figure, }
       \CommentTok{\#width = 8, height = 6, units = "in", dpi = 300)}

\CommentTok{\#Make a multi{-}panel figure}
\FunctionTok{grid.arrange}\NormalTok{(results\_figure\_1,  results\_figure\_2,  }\AttributeTok{nrow =} \DecValTok{2}\NormalTok{)}

\FunctionTok{grid.text}\NormalTok{(}\StringTok{"A"}\NormalTok{, }\AttributeTok{x =} \FloatTok{0.07}\NormalTok{, }\AttributeTok{y =} \FloatTok{0.95}\NormalTok{, }\AttributeTok{gp =} \FunctionTok{gpar}\NormalTok{(}\AttributeTok{fontsize =} \DecValTok{12}\NormalTok{)) }\CommentTok{\#Adds label \textquotesingle{}A\textquotesingle{} to first graph}
\FunctionTok{grid.text}\NormalTok{(}\StringTok{"B"}\NormalTok{, }\AttributeTok{x =} \FloatTok{0.57}\NormalTok{, }\AttributeTok{y =} \FloatTok{0.95}\NormalTok{, }\AttributeTok{gp =} \FunctionTok{gpar}\NormalTok{(}\AttributeTok{fontsize =} \DecValTok{12}\NormalTok{)) }\CommentTok{\#Adds label \textquotesingle{}B\textquotesingle{} to second graph}
\end{Highlighting}
\end{Shaded}

\includegraphics{PenguinAnalysis_files/figure-latex/result-figure-plots-1.pdf}
\emph{Figure 3. A plot \ldots{} }

\subsubsection{Conclusion}\label{conclusion}

Beak Shape Ratio: The ratio of culmen length to depth significantly
differs across species. Gentoo penguins have the largest ratio, followed
by Chinstrap penguins, and Adelie penguins have the smallest ratio.

Interpretation of Shape: Since the ratio is a measure of beak shape,
this suggests that Gentoo penguins have relatively longer, shallower
beaks, while Adelie penguins have shorter, deeper beaks.

Beak Length: Chinstrap and Gentoo penguins both have significantly
longer beaks than Adelie penguins. Gentoo penguins have slightly shorter
beaks than Chinstrap penguins, though this difference is still
statistically significant. Beak Depth: Gentoo penguins have
significantly deeper beaks compared to both Adelie and Chinstrap
penguins. There is no significant difference in beak depth between
Chinstrap and Adelie penguins. Are the results informative? Yes, these
results are informative because:

They show clear patterns of differences in beak morphology (both length
and depth) across species. The TukeyHSD results reveal specific
comparisons, quantifying the size and direction of differences between
species.

These findigns align well with feeding habits.

As the cliamte changes, wih rapidly decling extent and duration of sea
ice in Antartica, with correlated reductions in Antarctic krill6
(Euphausia superba), the main prey item for most meso- and top-predators
in the Antarctic ecosystem will decrease. A recent article suggetsed
that gentoo, which have the most diverse diets, will be able to adpat
whilst chinsatrp and adelie, have been showing recent declines in. As
such, nless these penguisna re able to adapt, they will perish (Clucas
et al., 2014). Signficantylm I argue that bill mropholgoy mught be amn
impritnat determinant of diet, as it will speciesl;ay how they hunt and
how ffectively they ctahc prey. of course mophological liamitaions and
gentics will limit how quicly these can adapt and diactet whetehr tbe
penguisna re able to survive at all.

\subsubsection{References}\label{references}

\ldots{}

\begin{center}\rule{0.5\linewidth}{0.5pt}\end{center}

\subsection{Open Science}\label{open-science}

\textbf{a) My GitHub link}

\url{https://github.com/Biology3579/ReproducibleScienceAssignment.git}

\textbf{b) Partner's GitHub link:}
\url{https://github.com/Anonymous94394/PalmerPenguins.git}

\subsubsection{c) Reflect on your experience running their code.
(300-500
words)}\label{c-reflect-on-your-experience-running-their-code.-300-500-words}

\begin{itemize}
\item
  \emph{What elements of your partner's code helped you to understand
  their data pipeline?}
\item
  \emph{Did it run? Did you need to fix anything?}
\item
  \emph{What suggestions would you make for improving their code to make
  it more understandable or reproducible, and why?}
\item
  \emph{If you needed to alter your partner's figure using their code,
  do you think that would be easy or difficult, and why?}
\end{itemize}

\subsubsection{d) Reflect on your own code based on your experience with
your partner's code and their review of yours. (300-500
words)}\label{d-reflect-on-your-own-code-based-on-your-experience-with-your-partners-code-and-their-review-of-yours.-300-500-words}

\begin{itemize}
\item
  \emph{What improvements did they suggest, and do you agree?}
\item
  \emph{What did you learn about writing code for other people?}
\end{itemize}

\end{document}
